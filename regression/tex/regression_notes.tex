%%%%%%%%%%%%%%%%%%%%%%%%%%%%%%%%%%%%%%%%%
% Programming/Coding Assignment
% LaTeX Template
%
% This template has been downloaded from:
% http://www.latextemplates.com
%
% Original author:
% Ted Pavlic (http://www.tedpavlic.com)
%
% Note:
% The \lipsum[#] commands throughout this template generate dummy text
% to fill the template out. These commands should all be removed when 
% writing assignment content.
%
% This template uses a Perl script as an example snippet of code, most other
% languages are also usable. Configure them in the "CODE INCLUSION 
% CONFIGURATION" section.
%
%%%%%%%%%%%%%%%%%%%%%%%%%%%%%%%%%%%%%%%%%

%----------------------------------------------------------------------------------------
%	PACKAGES AND OTHER DOCUMENT CONFIGURATIONS
%----------------------------------------------------------------------------------------

\documentclass{article}

\usepackage{fancyhdr} % Required for custom headers
\usepackage{lastpage} % Required to determine the last page for the footer
\usepackage{extramarks} % Required for headers and footers
\usepackage[usenames,dvipsnames]{color} % Required for custom colors
\usepackage{graphicx} % Required to insert images
\usepackage{listings} % Required for insertion of code
\usepackage{courier} % Required for the courier font
\usepackage{amsmath}
\usepackage{amssymb}
\usepackage{mathrsfs} % math script characters

% Margins
\topmargin=-0.45in
\evensidemargin=0in
\oddsidemargin=0in
\textwidth=6.5in
\textheight=9.0in
\headsep=0.25in

\linespread{1.1} % Line spacing

% Set up the header and footer
\pagestyle{fancy}
\lhead{\hmwkAuthorName} % Top left header
\chead{\hmwkTitle} % Top center head
\rhead{\firstxmark} % Top right header
\lfoot{\lastxmark} % Bottom left footer
\cfoot{} % Bottom center footer
\rfoot{Page\ \thepage\ of\ \protect\pageref{LastPage}} % Bottom right footer
\renewcommand\headrulewidth{0.4pt} % Size of the header rule
\renewcommand\footrulewidth{0.4pt} % Size of the footer rule

\setlength\parindent{0pt} % Removes all indentation from paragraphs

%----------------------------------------------------------------------------------------
%	CODE INCLUSION CONFIGURATION
%----------------------------------------------------------------------------------------

\definecolor{MyDarkGreen}{rgb}{0.0,0.4,0.0} % This is the color used for comments
\lstloadlanguages{Perl} % Load Perl syntax for listings, for a list of other languages supported see: ftp://ftp.tex.ac.uk/tex-archive/macros/latex/contrib/listings/listings.pdf
\lstset{language=Java, % Use Java in this example
        frame=single, % Single frame around code
        basicstyle=\small\ttfamily, % Use small true type font
        keywordstyle=[1]\color{Blue}\bf, % Perl functions bold and blue
        keywordstyle=[2]\color{Purple}, % Perl function arguments purple
        keywordstyle=[3]\color{Blue}\underbar, % Custom functions underlined and blue
        identifierstyle=, % Nothing special about identifiers                                         
        commentstyle=\usefont{T1}{pcr}{m}{sl}\color{MyDarkGreen}\small, % Comments small dark green courier font
        stringstyle=\color{Purple}, % Strings are purple
        showstringspaces=false, % Don't put marks in string spaces
        tabsize=5, % 5 spaces per tab
        %
        % Put standard Perl functions not included in the default language here
        morekeywords={rand},
        %
        % Put Perl function parameters here
        morekeywords=[2]{on, off, interp},
        %
        % Put user defined functions here
        morekeywords=[3]{test},
       	%
        morecomment=[l][\color{Blue}]{...}, % Line continuation (...) like blue comment
        numbers=left, % Line numbers on left
        firstnumber=1, % Line numbers start with line 1
        numberstyle=\tiny\color{Blue}, % Line numbers are blue and small
        stepnumber=5 % Line numbers go in steps of 5
}

% Creates a new command to include a java script, the first parameter is the filename of the script (without .pl), the second parameter is the caption
\newcommand{\javascript}[2]{
\begin{itemize}
\item[]\lstinputlisting[caption=#2,label=#1]{#1.java}
\end{itemize}
}

%----------------------------------------------------------------------------------------
%	DOCUMENT STRUCTURE COMMANDS
%	Skip this unless you know what you're doing
%----------------------------------------------------------------------------------------

% Header and footer for when a page split occurs within a problem environment
\newcommand{\enterProblemHeader}[1]{
\nobreak\extramarks{#1}{#1 continued on next page\ldots}\nobreak
\nobreak\extramarks{#1 (continued)}{#1 continued on next page\ldots}\nobreak
}

% Header and footer for when a page split occurs between problem environments
\newcommand{\exitProblemHeader}[1]{
\nobreak\extramarks{#1 (continued)}{#1 continued on next page\ldots}\nobreak
\nobreak\extramarks{#1}{}\nobreak
}

\setcounter{secnumdepth}{0} % Removes default section numbers
\newcounter{homeworkProblemCounter} % Creates a counter to keep track of the number of problems

\newcommand{\homeworkProblemName}{}
\newenvironment{homeworkProblem}[1][Problem \arabic{homeworkProblemCounter}]{ % Makes a new environment called homeworkProblem which takes 1 argument (custom name) but the default is "Problem #"
\stepcounter{homeworkProblemCounter} % Increase counter for number of problems
\renewcommand{\homeworkProblemName}{#1} % Assign \homeworkProblemName the name of the problem
\section{\homeworkProblemName} % Make a section in the document with the custom problem count
\enterProblemHeader{\homeworkProblemName} % Header and footer within the environment
}{
\exitProblemHeader{\homeworkProblemName} % Header and footer after the environment
}

\newcommand{\problemAnswer}[1]{ % Defines the problem answer command with the content as the only argument
\noindent\framebox[\columnwidth][c]{\begin{minipage}{0.98\columnwidth}#1\end{minipage}} % Makes the box around the problem answer and puts the content inside
}

\newcommand{\homeworkSectionName}{}
\newenvironment{homeworkSection}[1]{ % New environment for sections within homework problems, takes 1 argument - the name of the section
\renewcommand{\homeworkSectionName}{#1} % Assign \homeworkSectionName to the name of the section from the environment argument
\subsection{\homeworkSectionName} % Make a subsection with the custom name of the subsection
\enterProblemHeader{\homeworkProblemName\ [\homeworkSectionName]} % Header and footer within the environment
}{
\enterProblemHeader{\homeworkProblemName} % Header and footer after the environment
}

%----------------------------------------------------------------------------------------
%	NAME AND CLASS SECTION
%----------------------------------------------------------------------------------------

\newcommand{\hmwkTitle}{Regression\ Notes} % Assignment title


\newcommand{\hmwkAuthorName}{Ryan Gehring} % Your name

%----------------------------------------------------------------------------------------
%	TITLE PAGE
%----------------------------------------------------------------------------------------

\title{
\vspace{2in}
\textmd{\textbf{\hmwkTitle}}\\
\vspace{3in}
}

\author{\textbf{\hmwkAuthorName}}
\date{} % Insert date here if you want it to appear below your name

%----------------------------------------------------------------------------------------

\begin{document}

\maketitle

%----------------------------------------------------------------------------------------
%	TABLE OF CONTENTS
%----------------------------------------------------------------------------------------

%\setcounter{tocdepth}{1} % Uncomment this line if you don't want subsections listed in the ToC

\newpage
\tableofcontents
\newpage


%----------------------------------------------------------------------------------------
%	Derivation of Coefficient Estimates
%----------------------------------------------------------------------------------------


\begin{homeworkProblem}[Theory]

\subsection{Model Parametrization}

A vector of response variables $Y$ is estimated by the product of a matrix of
predictive variables $X$ and coefficient vector $\beta$ with normal errors $\epsilon$.

\begin{equation}
Y = X \beta + \epsilon
\end{equation}


\subsection{Loss Function}
Minimize the sum of squared errors.

\begin{equation}
SSE = \epsilon^T \epsilon
\end{equation}


\begin{equation*}
SSE = (Y - X\beta)^T (Y - X\beta)
\end{equation*}

\subsection{Normal Equation}

To minimize loss, differentiate with respect to the coefficients, set to zero: 
\begin{equation}
0 = -2X^T(Y - X\beta)
\end{equation}

\subsection{Coefficient Estimates}

Multiply by $(X^T X)^{-1}$ (and absorb 2 into $\beta$) to get the least squares estimate $\beta$:
\begin{equation}
\beta = (X^TX)^{-1}X^TY
\end{equation}




\end{homeworkProblem}



%----------------------------------------------------------------------------------------
%	Java Implementation
%----------------------------------------------------------------------------------------


% To have just one problem per page, simply put a \clearpage after each problem

\begin{homeworkProblem}[Application]
\subsection{Java Implementation}
See below for an implementation using Apache commons math. The Github Repo contains code to run the model against an arbitrary csv data file.

\javascript{/Users/ryangehring/proj/RyanProvesStuff/regression/java/regression_tools}{Regression using commons math}



\subsection{Estimation of Bear Age from Body Measurements}
Above code was used to generate predictions of bear ages from a variety of body measurements.

Bear Data (truncated for readability):

\begin{tabular}{ l l l l l l l l  }
  AGE & SEX & HEADLEN & HEADWTH & NECK & LENGTH & CHEST & WEIGHT \\ 
19 & 0 & 11 & 5.5 & 16 & 53 & 26 & 80 \\ 
55 & 0 & 16.5 & 9 & 28 & 67.5 & 45 & 344 \\ 
81 & 0 & 15.5 & 8 & 31 & 72 & 54 & 416 \\ 
115 & 0 & 17 & 10 & 31.5 & 72 & 49 & 348 \\ 
104 & 1 & 15.5 & 6.5 & 22 & 62 & 35 & 166 \\ 
100 & 1 & 13 & 7 & 21 & 70 & 41 & 220 \\ 
56 & 0 & 15 & 7.5 & 26.5 & 73.5 & 41 & 262 \\ 
51 & 0 & 13.5 & 8 & 27 & 68.5 & 49 & 360 \\ 
57 & 1 & 13.5 & 7 & 20 & 64 & 38 & 204 \\ 
53 & 1 & 12.5 & 6 & 18 & 58 & 31 & 144 \\ 
68 & 0 & 16 & 9 & 29 & 73 & 44 & 332 \\ 
8 & 0 & 9 & 4.5 & 13 & 37 & 19 & 34 \\ 
44 & 1 & 12.5 & 4.5 & 10.5 & 63 & 32 & 140 
\end{tabular}



$R^2$ was measured to be ~0.64.



\begin{tabular}{l l}
SEX:	& 26.148987054376104 \\
HEADLEN: & 3.0528248476353443 \\
HEADWTH: & 3.791600738790277 \\
NECK: & 1.366288821513322 \\
LENGTH: & -0.11350398628450656 \\
CHEST: & -2.9977072569525833 \\
WEIGHT: & 0.3122111457892039 
\end{tabular}



\end{homeworkProblem}



%----------------------------------------------------------------------------------------

\end{document}